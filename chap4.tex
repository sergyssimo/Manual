\chapter{Ajustando a Área de Tipo e o Layout da Página}
O pacote typearea oferece duas interfaces de usuário diferentes para influenciar a construção da área de tipo. O método mais importante é especificar opções ao carregar o pacote. Para informações sobre como configurar opções com o \KOMAScript\ consulte a seção 2.4. Nesta seção, as classes usadas nos exemplos não são classes \KOMAScript\ existentes, mas hipotéticas. Este guia assume que, idealmente, uma classe apropriada esteja disponível para cada tarefa.

\minisec{BCOR = correction}
Use a opção BCOR=correction para especificar o valor absoluto da correção de encadernação, ou seja, a largura da área perdida do papel durante o processo de encadernação. Este valor é então automaticamente levado em consideração ao construir o layout da página e é adicionado de volta à margem interna (ou esquerda) durante a saída. No valor da correção, você pode especificar qualquer unidade de medida entendida pelo TEX.

Ao usar uma classe \KOMAScript\ você não precisa carregar o pacote typearea explicitamente:
\begin{verbatim}
            \documentclass[BCOR=8.25mm]{scrreprt}
\end{verbatim}

Você pode omitir a opção a4paper com scrreprt, já que este é o padrão para todas as classes \KOMAScript.

Se você quiser definir a opção para um novo valor mais tarde, você pode, por exemplo, usar o seguinte:
\begin{verbatim}
            \documentclass{scrreprt}
            \KOMAoptions{BCOR=8.25mm}
\end{verbatim}

Os padrões são inicializados quando a classe scrreprt é carregada. Alterar uma configuração com os comandos \KOMAScript\ ou \verb|\KOMAoption| calculará automaticamente uma nova área de tipo com novas margens.
Note que você deve passar esta opção como uma opção de classe ao carregar uma das classes \KOMAScript\ como no exemplo acima, ou via \verb|\KOMAoptions| ou \verb|\KOMAoption| após carregar a classe. Quando você usa uma classe \KOMAScript\ você não deve carregar o pacote typearea explicitamente com \verb|\usepackage|, nem deve especificá-lo como um argumento opcional ao carregar o pacote se você estiver usando outra classe. Se a opção for alterada com \verb|\KOMAoptions| ou \verb|\KOMAoption| após carregar o pacote, a área de tipo e as margens são automaticamente recalculadas.

\minisec{DIV=factor}
A opção DIV=factor especifica o número de faixas nas quais a página é dividida horizontalmente e verticalmente durante a construção da área de texto. O método de construção exato é encontrado na seção 2.2. É importante perceber que quanto maior o fator, maior o bloco de texto e menores as margens. Qualquer valor inteiro maior que 4 é válido para fator.

Observe, no entanto, que valores grandes podem causar violações nas restrições nas margens da área de texto, dependendo de como você define outras opções. Em casos extremos, o cabeçalho pode cair fora da página. Ao usar a opção DIV=factor, você é responsável por cumprir com as restrições de margem e por escolher um comprimento de linha tipograficamente agradável.

Na tabela 2.1, você encontrará os tamanhos das áreas de texto para vários fatores DIV para a página A4 sem correção de encadernação. Nesse caso, as outras restrições que dependem do tamanho da fonte não são levadas em consideração.

Se for absolutamente necessário definir o texto com um espaçamento de linha de 1,5, não redefina \verb|\baselinestretch| em nenhuma circunstância. Embora esse procedimento seja recomendado com muita frequência, ele está obsoleto desde a introdução do \LaTeXe\ em 1994. No pior caso, use o comando \verb|\linespread|. Eu recomendo o pacote setspace, que não é parte do \KOMAScript. Você também deve deixar typearea recalcular uma nova área de tipo após alterar o espaçamento de linha. No entanto, você deve voltar ao espaçamento de linha normal para o título, e de preferência para o índice e várias listas — assim como a bibliografia e o índice. Para detalhes, veja a explicação de DIV=current.

Não raramente eu me pergunto por que eu me detenho em cálculos de typearea para um capítulo inteiro, quando seria muito mais fácil apenas fornecer um pacote com o qual você pode ajustar as margens como em um processador de texto. Muitas vezes é dito que tal pacote seria uma solução melhor em qualquer caso, já que todos sabem como escolher margens apropriadas, e que as margens calculadas pelo \KOMAScript\ não são tão boas assim. Eu gostaria de citar Hans Peter Willberg e Friedrich Forssmann, dois dos mais respeitados tipógrafos contemporâneos. (Você pode encontrar o original em alemão no guia alemão.):

\begin{quote}
 A prática de fazer as coisas por si mesmo é muito difundida, mas os resultados são frequentemente duvidosos porque tipógrafos amadores não veem o que está errado e não conseguem saber o que é importante. É assim que você se acostuma com tipografia incorreta e ruim. [\ldots] Agora, a objeção poderia ser feita de que tipografia é uma questão de gosto. Quando se trata de decoração, talvez se possa aceitar esse argumento, mas como tipografia é principalmente sobre informação, erros não só podem irritar, mas podem até mesmo causar danos.   
\end{quote}




