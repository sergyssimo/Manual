\chapter{Apêndice}
O apêndice de um documento consiste principalmente em suplementos ao documento. Partes típicas de um apêndice incluem uma bibliografia, um índice e um glossário. No entanto, você não deve iniciar um apêndice apenas para essas partes porque seu formato já as distingue do documento principal. Mas se houver elementos adicionais no apêndice, como documentos de terceiros citados, notas de rodapé, figuras ou tabulares, os elementos padrão, como a bibliografia, também devem fazer parte do apêndice.

\verb|\appendix|

O apêndice é iniciado nas classes padrão e \KOMAScript\ com \verb|\appendix|. Entre outras coisas, este comando altera a numeração dos capítulos para letras maiúsculas enquanto garante que as regras de acordo com [DUD96] para numerar os níveis de seccionamento sejam seguidas (para regiões de língua alemã). Essas regras são explicadas em mais detalhes na descrição da opção numbers na seção 3.16, página 99.

O formato dos títulos dos capítulos no apêndice é influenciado pelas opções chapterprefix e appendixprefix. Veja a seção 3.16, página 96 para mais informações.

Observe que \verb|\appendix| é um comando, não um ambiente! Este comando não espera um argumento. Capítulos e seções no apêndice usam \verb|\chapter| e \verb|\section|, assim como no texto principal.
