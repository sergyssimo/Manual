\chapter{Versos}
\begin{verbatim}
        \begin{verse}...\end{verse}    
\end{verbatim}

O ambiente de verso normalmente não é percebido como um ambiente de lista porque você não trabalha com comandos \verb|\item|. Em vez disso, quebras de linha fixas são usadas dentro do ambiente \texttt{flushleft}. Internamente, no entanto, tanto as classes padrão quanto o \KOMAScript\ implementam ele como um ambiente de lista.

Em geral, o ambiente de verso é usado para poesia. As linhas são recuadas tanto à esquerda quanto à direita. Linhas individuais de verso são finalizadas por uma quebra de linha fixa: \verb|\\|. Os versos são definidos como parágrafos, separados por uma linha vazia. Frequentemente, também \verb|\medskip| ou \verb|\bigskip| são usados. Para evitar uma quebra de página no final de uma linha de verso, você pode, como de costume, inserir \verb|\\*| em vez de \verb|\\|.

\textbf{Exemplo}: Como exemplo, o soneto de Emma Lazarus do pedestal de Liberty Enlightening the World:
\begin{verbatim}
\begin{verse}
Not like the brazen giant of Greek fame\\*
With conquering limbs astride from land to land\\*
Here at our sea-washed, sunset gates shall stand\\*
A mighty woman with a torch, whose flame\\*
Is the imprisoned lightning, and her name\\*
Mother of Exiles. From her beacon-hand\\*
Glows world-wide welcome; her mild eyes command\\*
The air-bridged harbor that twin cities frame.\\*
‘‘Keep, ancient lands, your storied pomp!’’ cries she\\*
With silent lips. ‘‘Give me your tired, your poor,\\*
Your huddled masses yearning to breathe free,\\*
The wretched refuse of your teeming shore.\\*
Send these, the homeless, tempest-tossed to me:\\*
I lift my lamp beside the golden door.’’
\end{verse} 
\end{verbatim}

No entanto, se você tiver linhas muito longas de verso onde uma quebra de linha ocorre dentro de uma linha de verso:
\newpage

\begin{verbatim}
\begin{verse}
Both the philosopher and the house-owner
always have something to repair.\\*
\bigskip
Don’t trust a man, my son, who tells you
that he has never lied.
\end{verse}
\end{verbatim}

