\chapter{Cabeçalhos e rodapés com scrlayer-scrpage}

Até a versão $3.11b$ do \KOMAScript\ o pacote \textbf{scrpage2} era a maneira recomendada de personalizar cabeçalhos e rodapés além das opções fornecidas pelos estilos de página headings, myheadings, plain e empty das classes \KOMAScript. Desde 2013, o pacote \textbf{scrlayer} foi incluído como um módulo básico do \KOMAScript. Este pacote fornece um modelo de camada e um novo modelo de estilo de página com base nele. No entanto, a interface do pacote é quase flexível demais e, consequentemente, não é fácil para o usuário médio compreender. Para obter mais informações sobre esta interface, consulte o capítulo 17 na parte II. No entanto, algumas das opções que realmente fazem parte do \textbf{scrlayer} e que, portanto, são retomadas naquele capítulo, também são documentadas aqui porque são necessárias para usar o \textbf{scrlayer-scrpage}.

Muitos usuários já estão familiarizados com os comandos do scrpage2. Por esse motivo, o scrlayer-scrpage fornece um método para manipular cabeçalhos e rodapés que é baseado no scrlayer, é amplamente compatível com o scrpage2 e, ao mesmo tempo, expande muito a interface do usuário. Se você já estiver familiarizado com o scrpage2 e se abster de chamadas diretas para seus comandos internos, normalmente pode usar o scrlayer-scrpage como um substituto imediato. Isso também se aplica à maioria dos exemplos usando o scrpage2 encontrados em livros do \LaTeX\ ou na Internet.

Além do scrlayer-scrpage ou scrpage2, você também pode usar o \textbf{fancyhdr} (veja [vO04]) para configurar os cabeçalhos e rodapés das páginas. No entanto, este pacote não tem suporte para vários recursos do \KOMAScript\ por exemplo, o esquema de elementos (veja \char`\\\texttt{set\-ko\-ma\-font}, \char`\\\texttt{add\-to\-ko\-ma\-font} e \char`\\\texttt{use\-ko\-ma\-font} na seção 3.6, página 58) ou o formato de numeração configurável para cabeçalhos dinâmicos (veja a opção numbers e, por exemplo, \char`\\\texttt{chap\-ter\-mark\-for\-mat} na seção 3.16, página 99 e página 112). Portanto, se você estiver usando uma classe \KOMAScript\ você deve usar o novo pacote scrlayer-scrpage. Se você tiver problemas, você ainda pode usar o scrpage2. Claro, você também pode usar o scrlayer-scrpage com outras classes, como as do \LaTeX\ padrão.

Além dos recursos descritos neste capítulo, o scrlayer-scrpage fornece várias outras funções que provavelmente interessam apenas a um número muito pequeno de usuários e, portanto, são descritas no capítulo 18 da parte II, começando na página 448. No entanto, se as opções descritas na parte I forem insuficientes para seus propósitos, você deve examinar o capítulo 18.


